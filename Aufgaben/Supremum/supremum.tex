\documentclass{article}
\usepackage{german}
\usepackage[latin1]{inputenc}
\usepackage{a4wide}
\usepackage{amssymb}
\usepackage{fancyvrb}
\usepackage{alltt}
\usepackage{fleqn}
\usepackage{theorem}

{\theorembodyfont{\sf}
\newtheorem{Definition}{Definition}
}

\newcommand{\solution}{\vspace*{0.2cm}

\noindent
\textbf{L�sung}: }

\def\pair(#1,#2){\langle #1, #2 \rangle}

\newcommand{\qed}{\hspace*{\fill} $\Box$
\vspace*{0.2cm}

}

\newcommand{\eod}{\hspace*{\fill} $\diamond$
\vspace*{0.2cm}

}

\newcommand{\eox}{\hspace*{\fill} $\diamond$
\vspace*{0.2cm}

}

\newcounter{aufgabe}
\newcommand{\exercise}{\vspace*{0.4cm}
\stepcounter{aufgabe}

\noindent
\textbf{Aufgabe \arabic{aufgabe}}: }

\begin{document}
\noindent
Zur Vereinfachung der Schreibweise vereinbaren wir f�r eine Menge $X \subseteq D$ und ein Element 
$a \in D$ die folgenden Kurzschreibweisen:
\begin{enumerate}
\item $X \leq a \;\stackrel{\mathrm{def}}{\Longleftrightarrow}\; \forall x \in X: x \leq a$,
\item $a \leq X \;\stackrel{\mathrm{def}}{\Longleftrightarrow}\; \forall x \in X: a \leq x$. 
\end{enumerate}

\begin{Definition}[Infimum, vollst�ndige Ordnung]
Es sei $\pair(D, \leq)$ eine lineare Ordnung.
Eine Menge $X \subseteq D$ ist \emph{nach unten beschr�nkt}, falls es ein
 $u \in D$ gibt, so dass 
\\[0.2cm]
\hspace*{1.3cm}
$u \leq X$.
\\[0.2cm]
Ein $i \in D$ ist das \emph{Infimum} einer Menge $X$, wenn $i$ die gr��te
untere Schranke von $X$ ist, wenn also 
\\[0.2cm]
\hspace*{1.3cm}
$i \leq X$ \quad \mbox{und} \quad
$\forall u \in D: u \leq X \rightarrow u \leq i$
\\[0.2cm]
gilt.  In diesem Fall schreiben wir
\\[0.2cm]
\hspace*{1.3cm}
$i = \inf(X)$.
\\[0.2cm]
Eine lineare Ordnung $\langle D, \leq \rangle$ ist eine \emph{vollst�ndige} Ordnung, wenn
jede nicht-leere Menge $X \subseteq D$, die nach unten beschr�nkt ist, ein Infimum hat.
\end{Definition}

\begin{Definition}[Supremum]
Es sei $\pair(D, \leq)$ eine lineare Ordnung.
Eine Menge $X \subseteq D$ ist \emph{nach oben beschr�nkt}, falls es ein
 $o \in D$ gibt, so dass 
\\[0.2cm]
\hspace*{1.3cm}
$X \leq o$.
\\[0.2cm]
Ein $s \in D$ ist das \emph{Supremum} einer Menge $X$, wenn $s$ die kleinste
obere Schranke von $X$ ist, wenn also 
\\[0.2cm]
\hspace*{1.3cm}
$X \leq s$ \quad \mbox{und} \quad
$\forall o \in D: X \leq o \rightarrow s \leq o$
\\[0.2cm]
gilt.  In diesem Fall schreiben wir
\\[0.2cm]
\hspace*{1.3cm}
$i = \sup(X)$.
\end{Definition}
\pagebreak



\exercise
Es sei $\pair(D, \leq)$ eine vollst�ndige Ordnung.  Die Menge $X \subseteq D$ sei nicht leer und nach oben beschr�nkt.
Zeigen Sie, dass $X$ dann ein Supremum besitzt. \eox

\solution
Wir definieren nun $O$ als die Menge der oberen Schranken von $X$:
\\[0.2cm]
\hspace*{1.3cm}
$O := \{ o \in D \mid X \leq o \}$.
\\[0.2cm]
Da $X$ nach oben beschr�nkt ist, ist $O$ sicher nicht leer.  Da $X$ nicht leer ist, gibt es ein 
$x_0 \in X$ und dann gilt
\\[0.2cm]
\hspace*{1.3cm}
$x_0 \leq O$.
\\[0.2cm]
Folglich ist die Menge $O$ durch $x_0$ nach unten beschr�nkt.  Da $O$ au�erdem nicht leer ist, hat
$O$ dann ein Infimum, denn $\pair(D, \leq)$ ist eine vollst�ndige Ordnung.  Wir definieren
\\[0.2cm]
\hspace*{1.3cm}
$s := \inf(O)$.
\\[0.2cm]
Wir werden zeigen, dass auch
\\[0.2cm]
\hspace*{1.3cm}
$s = \sup(X)$
\\[0.2cm]
gilt.  Dazu sind zwei Bedingungen nachzuweisen.
\begin{enumerate}
\item Wir zeigen: $s$ ist eine obere Schranke von $X$.  Dazu ist $X \leq s$ nachzuweisen.

      Sei also $x \in X$.  Zu zeigen ist $x \leq s$.  Wir f�hren diesen Nachweis
      indirekt und nehmen $s < x$ an.   Aus 
      $s < x$ und $s = \inf(O)$ folgt, dass $x$ keine untere Schranke von $O$ sein kann, denn $s$
      ist ja die gr��te untere Schranke von $O$.  Wir haben also 
      \\[0.2cm]
      \hspace*{1.3cm}
      $\neg (x \leq O)$.
      \\[0.2cm]
      Folglich existiert ein $o \in O$ mit $o < x$.  Da f�r alle $o \in O$ nach Definition
      der Menge $O$ als der Menge der oberen Schranken von $X$ die Ungleichung
      \\[0.2cm]
      \hspace*{1.3cm}
      $x \leq o$
      \\[0.2cm]
      gilt, haben wir einen Widerspruch zu der Annahme $s < x$ und folglich muss
      $x \leq s$ gelten.
\item Wir zeigen: $s$ ist die kleinste obere Schranke von $X$.      

      Wir nehmen an, dass $o$ eine weitere obere Schranke von $X$ ist.  Wir m�ssen $s \leq o$
      zeigen.  Dann gilt offenbar $o \in O$.  Da $s$ als das Infimum von $O$ definiert ist, folgt
      ist, gilt $s \leq O$ und daraus folgt sofort $s \leq o$. \qed
\end{enumerate}
\end{document}

%%% Local Variables: 
%%% mode: latex
%%% TeX-master: t
%%% End: 
