\documentclass{article}
\usepackage{german}
\usepackage[latin1]{inputenc}
\usepackage{a4wide}
\usepackage{epsfig}
\usepackage{amssymb}
\usepackage{fancyvrb}
\usepackage{alltt}
\usepackage{fleqn}
\usepackage{epic}

\newtheorem{Definition}{Definition}
\newtheorem{Axiom}[Definition]{Axiom}
\newtheorem{Notation}[Definition]{Notation}
\newtheorem{Korollar}[Definition]{Korollar}
\newtheorem{Lemma}[Definition]{Lemma}
\newtheorem{Satz}[Definition]{Satz}
\newtheorem{Theorem}[Definition]{Theorem}

\renewcommand{\labelenumi}{(\alph{enumi})}

\title{Analysis}
\author{Karl Stroetmann}

\newcommand{\bruch}[2]{\displaystyle\frac{\;\displaystyle#1\;}{\;\displaystyle#2\;}}
\newcommand{\folge}[1]{\bigl(#1\bigr)_{n\in\mathbb{N}}}
\newcommand{\Folge}[1]{\left(#1\right)_{n\in\mathbb{N}}}
\newcommand{\Reihe}[1]{\left(\sum\limits_{i=0}^n #1\right)_{n\in\mathbb{N}}}
\newcommand{\Oh}{\mathcal{O}}
\newcommand{\df}{\displaystyle\frac{d\;}{dx}}
\newcommand{\err}[1]{\textsl{error}_n(#1)}
\newcommand{\erri}[2]{\textsl{error}^{(#2)}_n(#1)}
\newcommand{\norm}[1]{\big\|#1\bigr\|_{\infty}}

\def\pair(#1,#2){\langle #1, #2 \rangle}

\newlength{\mylength}
\setlength{\mathindent}{1.3cm}


\begin{document}
\noindent
\section{Taylor-Reihe}
\textbf{Aufgabe}: Leiten Sie die folgende Formel aus dem Additions-Theorem des
Arcus-Tangens her und berechnen Sie damit $\pi$ auf eine Genauigkeit von $10^{-9}$:
\begin{equation}
  \label{eq:Pi}
  \bruch{\pi}{4} =  2 * \arctan\Bigl(\frac{1}{2}\Bigr) -\arctan\Bigl(\frac{1}{7}\Bigr).  
\end{equation}

\noindent
\textbf{L\"osung}:
Zum Beweis der oben angegebenen Gleichung ben\"otigen wir zwei
Gleichungen.  Einerseits gilt $\arctan(1) = \frac{\pi}{4}$, andererseits haben wir das
Additions-Theorem f\"ur den Arcus-Tangens: 
\\[0.3cm]
\hspace*{1.3cm}
$\displaystyle \arctan(x) + \arctan(y) = \arctan\Bigl(\frac{x + y}{1 - x*y}\Bigr)$
\\[0.3cm]
Daher haben wir
\\[0.3cm]
\hspace*{1.3cm}
$
\begin{array}[t]{lrcl}
                 & \bruch{\pi}{4}  & = & 2 * \arctan\Bigl(\frac{1}{2}\Bigr) -\arctan\Bigl(\frac{1}{7}\Bigr) \\[0.3cm]
 \Leftrightarrow & \arctan(1)      & = & 2 * \arctan\Bigl(\frac{1}{2}\Bigr) -\arctan\Bigl(\frac{1}{7}\Bigr) \\[0.3cm]
 \Leftrightarrow & \arctan(1) + \arctan\Bigl(\frac{1}{7}\Bigr) & = & \arctan\Bigl(\frac{1}{2}\Bigr) + \arctan\Bigl(\frac{1}{2}\Bigr) \\[0.3cm]
 \Leftrightarrow & \displaystyle \arctan\left( \frac{1 + \frac{1}{7}}{1 - \frac{1}{7}}\right) & = & 
                   \displaystyle \arctan\left( \frac{\frac{1}{2} + \frac{1}{2}}{1 - \frac{1}{2}*\frac{1}{2}}\right) \\[0.5cm]
 \Leftrightarrow & \displaystyle \arctan\left( \frac{\frac{8}{7}}{\frac{6}{7}}\right) & = & 
                   \displaystyle \arctan\left( \frac{1}{\frac{3}{4}}\right) \\[0.5cm]
 \Leftrightarrow & \displaystyle \arctan\left( \frac{8}{6} \right) & = & 
                   \displaystyle \arctan\left( \frac{4}{3} \right) \quad\surd\\[0.3cm]
\end{array}
$
\\[0.3cm]
Wir berechnen $\pi$ nun indem wir Gleichung (\ref{eq:Pi}) mit $4$ multizieren und den
Arcus-Tangens durch eine Taylor-Reihe approximieren.  Das liefert die Formel
\begin{equation}
  \label{eq:PiExact}  
\pi = 8 * \sum\limits_{k=0}^\infty \bruch{(-1)^k}{2*k+1} * \left(\bruch{1}{2}\right)^{2*k+1} 
      - 4 * \sum\limits_{k=0}^\infty \bruch{(-1)^k}{2*k+1} * \left(\bruch{1}{7}\right)^{2*k+1}
\end{equation}
In der Praxis k\"onnen wir keine unendliche Summation durchf\"uhren,  wir m\"ussen die Summation
irgendwann abbrechen.  Die N\"aherung, die wir berechnen, hat also die Form
\begin{equation}
  \label{eq:PiApprox}
\pi \approx 8 * \sum\limits_{k=0}^n \bruch{(-1)^k}{2*k+1} * \left(\bruch{1}{2}\right)^{2*k+1} 
          - 4 * \sum\limits_{k=0}^n \bruch{(-1)^k}{2*k+1} * \left(\bruch{1}{7}\right)^{2*k+1}
\end{equation}
Der \emph{Abbruch-Fehler} $e$ misst die Differenze zwischen der durch Gleichung (\ref{eq:PiApprox})
gegebenen N\"aherung und dem durch Gleichung (\ref{eq:PiExact}) gegebenen exakten Wert, es
gilt also 
\\[0.3cm]
\hspace*{1.3cm}
$
\begin{array}[t]{lcl}
e & = & \displaystyle \pi - \left( 8 * \sum\limits_{k=0}^n \bruch{(-1)^k}{2*k+1} * \left(\bruch{1}{2}\right)^{2*k+1} 
              - 4 * \sum\limits_{k=0}^n \bruch{(-1)^k}{2*k+1} * \left(\bruch{1}{7}\right)^{2*k+1} \right)\\[0.6cm]
    & = & \displaystyle 8 * \sum\limits_{k=0}^\infty \bruch{(-1)^k}{2*k+1} * \left(\bruch{1}{2}\right)^{2*k+1} 
          - 4 * \sum\limits_{k=0}^\infty \bruch{(-1)^k}{2*k+1} * \left(\bruch{1}{7}\right)^{2*k+1} \\[0.5cm]
    &   & \displaystyle - \left( 8 * \sum\limits_{k=0}^n \bruch{(-1)^k}{2*k+1} * \left(\bruch{1}{2}\right)^{2*k+1} 
                     - 4 * \sum\limits_{k=0}^n \bruch{(-1)^k}{2*k+1} * \left(\bruch{1}{7}\right)^{2*k+1}\right) \\[0.5cm]
    & = & \displaystyle 8 * \sum\limits_{k=n+1}^\infty \bruch{(-1)^k}{2*k+1} * \left(\bruch{1}{2}\right)^{2*k+1} 
          - 4 * \sum\limits_{k=n+1}^\infty \bruch{(-1)^k}{2*k+1} * \left(\bruch{1}{7}\right)^{2*k+1}.
\end{array}
$
\\[0.3cm]
Damit k\"onnen wir den Betrag des Abbruch-Fehlers wie folgt absch\"atzen:
\\[0.3cm]
\hspace*{0.3cm}
$
\begin{array}[t]{lcl}
|e|    & \leq  & \displaystyle 8 * \left| \sum\limits_{k=n+1}^\infty \bruch{(-1)^k}{2*k+1} * \left(\bruch{1}{2}\right)^{2*k+1} \right|
               \;+\; 4 * \left| \sum\limits_{k=n+1}^\infty \bruch{(-1)^k}{2*k+1} * \left(\bruch{1}{7}\right)^{2*k+1}\right| \\[0.6cm]
   & \leq  & \displaystyle 8 * \left| \bruch{(-1)^{n+1}}{2*(n+1)+1} * \left(\bruch{1}{2}\right)^{2*(n+1)+1} \right|
               \;+\; 4 * \left| \bruch{(-1)^{n+1}}{2*(n+1)+1} * \left(\bruch{1}{7}\right)^{2*(n+1)+1}\right| \\[0.6cm]
   & \leq  & \displaystyle 8 * \left(\bruch{1}{2}\right)^{2*n+3} 
               + 4 *  \left(\bruch{1}{2}\right)^{2*n+3}  \\[0.5cm]
   & \leq  & \displaystyle 12 * \left(\bruch{1}{2}\right)^{2*n+3}. \\[0.5cm]
\end{array}
$
\\[0.1cm]
Dabei haben wir im ersten Schritt dieser Absch\"atzung das Kriterium von Leibniz
angewendet:  Der Betrag einer alternierenden Reihe ist kleiner als das erste Glied dieser
Reihe, also gilt
\\[0.3cm]
\hspace*{1.3cm}
$\left|\sum\limits_{k=0}^\infty (-1)^n * a_n \right| \leq |a_0|$.
\\[0.3cm]
Wollen wir jetzt $\pi$ auf eine Genauigkeit von $10^{-9}$ berechnen, so m\"ussen wir $n$ so
gro\3 w\"ahlen, dass gilt:
\\[0.1cm]
\hspace*{0.3cm}
$
\begin{array}[t]{lrcl}
            & |e| & \leq & 10^{-9} \\[0.1cm]
\Leftarrow & \displaystyle 12 * \left(\bruch{1}{2}\right)^{2*n+3} & \leq & 10^{-9} \\[0.3cm]
\Leftrightarrow & \displaystyle \ln(12) - (2*n+3)*\ln(2) & \leq & -9* \ln(10) \\[0.5cm]
\Leftrightarrow & \displaystyle - (2*n+3)*\ln(2) & \leq & -9* \ln(10) - \ln(12) \\[0.5cm]
\Leftrightarrow & \displaystyle   2*n+3 & \geq & \bruch{9* \ln(10) + \ln(12)}{\ln(2)} \\[0.5cm]
\Leftrightarrow & \displaystyle   n & \geq & \bruch{1}{2}*\left(\bruch{9* \ln(10) + \ln(12)}{\ln(2)} - 3 \right) \approx 15.24115768 \\[0.5cm]
\Leftarrow & n & = & 16.
\end{array}
$
\\[0.3cm]
Fall wir also die ersten $16$ Glieder der Reihen ber\"ucksichtigen, k\"onnen wir $\pi$ auf
eine Genauigkeit von $10^{-9}$ berechnen.  Rechnen wir mit 20 Stellen hinter dem Komma, so
erhalten wir den Wert 
\\[0.1cm]
\hspace*{1.3cm}
$
\begin{array}[t]{llcl}
                    & \pi & \approx & 3.1415926535951738312. \\[0.1cm]
  \mbox{exakt gilt} & \pi & =       & 3.1415926535897932\cdots. \\
\end{array}
$
\\[0.1cm]
Wir haben also $\pi$ auf eine Genauigkeit von 10 Stellen hinter dem Komma berechnet.

\section{Interpolation}
\textbf{Aufgabe}: F\"ur die Funktion $x \mapsto \sin(x)$ soll im Intervall
$[0,\frac{\pi}{2}]$ ein Tabelle erstellt werden, so dass der bei linearer Interpolation entstehende
Interpolations-Fehler kleiner als $10^{-5}$ ist.  Das Intervall $[0,\frac{\pi}{2}]$ soll zu diesem
Zweck in gleich gro\3e Intervalle aufgeteilt werden.  Berechnen Sie die Anzahl der
Eintr\"age, die f\"ur die Erstellung der Tabelle notwendig ist.
\vspace*{0.3cm}

\noindent
\textbf{L\"osung}:  
Wir untersuchen den Interpolations-Fehler, der innerhalb eines beliebiegen Intervalls 
$[x_0, x_1] \subseteq [0,\frac{\pi}{2}]$ auftritt.  Die L\"ange dieses Intervalls ist $h := x_1 - x_0$.
Bei linearer Interploation gilt f\"ur den Interpolations-Fehler
\\[0.1cm]
\hspace*{1.3cm}
$e(x) := \sin(x) - p(x) = \bruch{\sin^{(2)}(\xi)}{2!} * (x-x_0)*(x - x_1)$ \quad mit $\xi\in [x_0,x_1]$,
\\[0.3cm]
denn bei linearer Interploation gilt $n=1$ und folglich $n+1 = 2$.
Da die zweite Ableitung von $x \mapsto \sin(x)$ die Funktion $x \mapsto -\sin(x)$ ist und
da weiter $|\sin(x)| \leq 1$ ist, k\"onnen wir f\"ur den Betrag des Interpolations-Fehlers 
schreiben
\begin{equation}
  \label{eq:interpolationSinus}
|e(x)| \leq \bruch{1}{2} * |(x-x_0)*(x-x_1)|. 
\end{equation}
Wir definieren 
\\[0.1cm]
\hspace*{1.3cm} $w(x) := (x-x_0)*(x-x_1)$. 
\\[0.1cm]
Um $|e(x)|$ n\"aher bestimmen zu k\"onnen, m\"ussen wir die Funktion $w(x)$ in dem Intervall
$[x_0,x_1]$ untersuchen.  Insbesondere interessiert uns die Frage, wo die Funktion $w(x)$
in dem Intervall $[x_0,x_1]$ Extremwerte (also Minima oder Maxima) annimmt.  Falls
die Funktion $w(x)$ an der Stelle $\chi$ einen  Extremwert hat, dann muss die Ableitung
$\df w(\chi)$ den Wert 0 haben: 
\\[0.1cm]
\hspace*{1.3cm}
$\df w(\chi) = 0$.
\\[0.1cm]
Es gilt 
\\[0.1cm]
\hspace*{1.3cm}
$\df w(x) = \df \Bigl( x^2 - (x_0 + x_1) * x + x_0 * x_1\Bigr) = 2*x - (x_0 + x_1)$.
\\[0.3cm]
F\"ur einen Extremwert im Innern des Intervalls $[x_0,x_1]$ muss also gelten:
\\[0.3cm]
\hspace*{1.3cm}
$0 = 2 * \chi - (x_0 + x_1)$, \quad also \quad $\chi = \bruch{x_0 + x_1}{2}$.
\\[0.3cm]
Falls die Funktion $w(x) = (x - x_0)*(x - x_1)$ in dem Intervall einen Extremwert annimmt,
so liegt dieser entweder an der Stelle $\chi = \frac{1}{2}(x_0 + x_1)$ oder an den
Intervall-Grenzen $x_0$ bzw.~$x_1$.  Setzen wir diese Punkte ein, so erhalten wir 
\\[0.1cm]
\hspace*{1.3cm}
$w(x_0) = (x_0 - x_0)*(x_1-x_0) = 0$, 
\\[0.1cm]
\hspace*{1.3cm}
$w(x_1) = (x_1 - x_0)*(x_1-x_1) = 0$, \quad und 
\\[0.1cm]
\hspace*{1.15cm}
$
\begin{array}[t]{lcl}
w\Bigl(\frac{1}{2}(x_0 + x_1)\Bigr) & = & \left(\frac{1}{2}(x_0 + x_1) - x_0\right)*\left(\frac{1}{2}(x_0 + x_1)-x_1\right) \\[0.3cm]
                                    & = & \left(\frac{1}{2}(x_1 - x_0)\right)*\left(\frac{1}{2}(x_0 - x_1)\right) \\[0.3cm]
                                    & = & - \frac{1}{4} (x_1 - x_0)^2 \\[0.3cm]
                                    & = & - \frac{1}{4} h^2. \\[0.3cm]
\end{array}\\[0.1cm]
$
\\[0.3cm]
Damit ist klar, dass wir in Gleichung (\ref{eq:interpolationSinus}) den Term
$\bigl|(x-x_0)*(x-x_1)\bigr|$ durch $\frac{h^2}{4}$ absch\"atzen k\"onnen, wir haben also f\"ur
den Interpolations-Fehler die Absch\"atzung 
\begin{equation}
  \label{eq:interpolationSinus2}
  |e(x)| = \bigl|\sin(x) -p(x)\bigr| \leq \bruch{h^2}{8}
\end{equation}
gefunden.   Wir erreichen bei linearer Interploation eine Genauigkeit von
$10^{-5}$ falls wir $h$ so w\"ahlen, dass $|e(x)| < 10^{-5}$ gilt.  Also bestimmen wir $h$
wie folgt: 
\\[0.1cm]
\hspace*{1.3cm}
$\bruch{h^2}{8} \leq 10^{-5} \;\Leftrightarrow\; h \leq \sqrt{8 * 10^{-5}} \approx 0.008944271908$.
\\[0.1cm]
Wegen \\[0.1cm]
\hspace*{1.3cm}
$\bruch{\pi}{2h} \approx 175.6203682$
\\[0.1cm] 
m\"ussen wir das Intervall $[0,\frac{\pi}{2}]$ also in 176 Intervalle aufteilen.
Die Tabelle hat dann 177 Eintr\"age.
\pagebreak

\noindent
\textbf{Aufgabe}: L\"osen Sie f\"ur $y=10^6$ und $y=10^{-6}$ die Gleichung $x*\exp(x) = y$ 
durch eine einfache Fixpunkt-Iteration.  Berechnen Sie die L\"osung $x$ jeweils auf eine
Genauigkeit von $10^{-3}$.
\vspace*{0.3cm}

\noindent
\textbf{L\"osung}:
Es gibt  zwei auf der Hand liegende M\"oglickeiten um die Gleichung $x*\exp(x) = y$ in eine
Fixpunkt-Gleichung zu verwandeln.  
\begin{enumerate}
\item $x*\exp(x) = y \;\Leftrightarrow\; x = y*\exp(-x)$.

      In diesem Fall definieren wir $f(x) := y*\exp(-x)$. Dann gilt
      \\[0.1cm]
      \hspace*{1.3cm} $f'(x) = -y*\exp(-x)$.
      \\[0.1cm]
      Da $x>0$ sein muss, k\"onnen wir den Betrag $|f'(x)|$ durch $y$ absch\"atzen: 
      \\[0.1cm]
      \hspace*{1.3cm}
      $|f'(x)| \leq y$
      \\[0.1cm]
      Falls nun $y$ klein ist, wird die Iteration $x_{n+1}=f(x_n)$ schnell konvergieren,
      in dem konkreten Fall $y=10^{-6}$ ist $f$ kontrahierend mit dem
      Kontraktions-Koeffizienten $10^{-6}$.  W\"ahlen wir als Startwert $x_0 = 0$,
      so erhalten wir $x_1 = 10^{-6}$ und mit dem Banach'schen Fixpunkt-Satz erhalten wir
      als Absch\"atzung f\"ur den Fehler nach $n$-Iterationen 
      \\[0.1cm]
      \hspace*{1.3cm}
      $|x_n - \bar{x}| \leq \bruch{\bigl(10^{-6}\bigr)^n}{1-10^{-6}}*|x_1-x_0| \approx
      10^{-6\cdot n} *|10^{-6}| = 10^{-6\cdot (n+1)}$ 
      \\[0.1cm]
      Setzen wir hier $n=1$ so sehen wir 
      \\[0.1cm]
      \hspace*{1.3cm}
      $|x_1 - \bar{x}| \leq  10^{-12}$.
      \\[0.1cm]
      Damit ist die geforderte Genauigkeit bereits nach einer Iteration erreicht
      und der gesuchte Wert ist $\bar{x} \approx 10^{-6}$.
\item $x*\exp(x) = y \;\Leftrightarrow\;  x = \ln(y) - \ln(x)$.

      In diesem Fall definieren wir $f(x) := \ln(y) - \ln(x)$.  Dann gilt 
      \\[0.1cm]
      \hspace*{1.3cm}
      $f'(x) = -\bruch{1}{x}$.
      \\[0.1cm]
      Diese Funktion ist auf den Intervallen kontrahierend, auf denen $x$ gr\"o\3er als 1
      ist. Sie ist geeignet, die L\"osung der Gleichung f\"ur $y= 10^6$ zu finden. 
      Setzen wir $x_0 := 1$, so erhalten wir die Werte 
      \\[0.1cm]
      \hspace*{0.3cm}
      $x_1 = 13.815510$, \quad $x_2 = 11.18971$, \quad $x_3 = 11.40051$, \quad
      $x_4 = 11.38185$, \quad $x_5 = 11.38349$.
      \\[0.1cm]
      Wir vermuten, dass die geforderte Genauigkeit im f\"unften Schritt erreicht ist.  Nach
      dem ersten Schritt sind alle Werte von $x$ jedenfalls gr\"o\3er als 10, so dass wir den
      Kontraktions-Koeffizienten $q$ durch $\frac{1}{10}$ absch\"atzen k\"onnen. In der Vorlesung
      wurde allgemein gezeigt, dass 
      \\[0.1cm]
      \hspace*{1.3cm}
      $|x_{n} - \bar{x}| \leq \bruch{q}{1-q}*|x_n - x_{n-1}|$
      \\[0.1cm] gilt. Setzen wir die konkreten Werte ein, so finden wir
      \\[0.1cm]
      \hspace*{0.3cm}
      $|x_5 - \bar{x}| \leq \bruch{q}{1-q} * |x_5 - x_4| \approx
      \frac{\frac{1}{10}}{\frac{9}{10}}*|11.38185 - 11.38349| \approx
      \bruch{1}{9}*0.00163 \approx 1.8*10^{-4}$
      \\[0.1cm]
      Damit hat $x_5$ tats\"achlich die geforderte Genauigkeit.
\end{enumerate}
\pagebreak

\noindent
\textbf{Aufgabe}: Untersuchen Sie mit Hilfe des Integral-Vergleichskriteriums, ob die
Reihe 
\\[0.1cm]
\hspace*{1.3cm}
$\displaystyle \sum\limits_{n=1}^\infty \bruch{1}{n*(n+1)}$ 
\\[0.3cm]
konvergiert.
\vspace*{0.1cm}

\noindent
\textbf{L\"osung}:  Nach dem Integral-Vergleichskriterium konvergiert die Reihe genau dann,
wenn der Grenzwert
\\[0.1cm]
\hspace*{1.3cm}
$\displaystyle \lim\limits_{x \rightarrow \infty} \int_1^x \bruch{1}{t\cdot(t+1)}\,dt$
\\[0.1cm]
existiert.  Um das Integral ausf\"uhren zu k\"onnen, f\"uhren wir eine Partialbruch-Zerlegung
des Integranden durch:
\\[0.1cm]
\hspace*{1.3cm}
$
\begin{array}[t]{llcl}
                & \bruch{1}{t\cdot(t+1)} & = & \bruch{\alpha}{t} + \bruch{\beta}{t+1} \\[0.3cm]
\Leftrightarrow & 1 & = & \alpha \cdot (t+1) + \beta \cdot t \\[0.1cm]
\Leftrightarrow & 1 & = & t\cdot (\alpha + \beta) + \alpha   \\[0.1cm]
\end{array}
$
\\[0.1cm]
Damit haben wir zur Bestimmung von $\alpha$ und $\beta$ die Gleichungen 
\\[0.1cm]
\hspace*{1.3cm}
$\alpha = 1$ \quad und \quad $0 = \alpha + \beta$, \quad also \quad $\beta = -1$
\\[0.1cm]
gefunden und damit gilt 
\\[0.1cm]
\hspace*{1.3cm}
$
\begin{array}[t]{clcl}
    & \displaystyle \lim\limits_{x \rightarrow \infty} \int_1^x \bruch{1}{t\cdot(t+1)}\,dt 
& = & \displaystyle \lim\limits_{x \rightarrow \infty} \int_1^x \Bigl(\bruch{1}{t} - \bruch{1}{t+1}\Bigr)\,dt \\[0.3cm]
  = & \displaystyle \lim\limits_{x \rightarrow \infty} \Bigl(\ln(t) - \ln(t+1)\Bigr)\Big|_1^x 
& = & \displaystyle \lim\limits_{x \rightarrow \infty} \ln\Bigl(\bruch{t}{t+1}\Bigr)\Big|_1^x \\[0.3cm]
  = & \displaystyle \lim\limits_{x \rightarrow \infty} \ln\Bigl(\bruch{x}{x+1}\Bigr) - \ln\Bigl(\bruch{1}{2}\Bigr)  
& = & \displaystyle \lim\limits_{x \rightarrow \infty} \ln\biggl(\frac{1}{1+\frac{1}{x}}\biggr) + \ln(2)  \\[0.3cm]
  = & \displaystyle \ln\left(\bruch{1}{1+0}\right) + \ln(2) 
& = & \displaystyle \ln(1) + \ln(2) \\[0.3cm]
  = & \ln(2).
\end{array}
$
\\[0.1cm]
Da der Grenzwert existiert, konvergiert die obige Reihe.
\pagebreak

\noindent
\textbf{Aufgabe}:  Berechnen Sie, in wieviele Teil-Intervalle das Intervall $[0,1]$
aufgeteilt werden muss, wenn das Integral
\\[0.1cm]
\hspace*{1.3cm}
$\displaystyle \int_0^1 e^{-x^2}\, dx$
\\[0.1cm]
mit Hilfe der Trapez-Regel mit einer Genauigkeit von $10^{-6}$ berechnet werden soll.
\vspace*{0.3cm}

\noindent
Nach dem Satz \"uber die Fehlerabsch\"atzung bei der Trapez-Regel kann der Fehler, der bei der
Approximation des Integrals mit der Trapez-Regel entsteht, durch den Ausdruck 
\\[0.3cm]
\hspace*{1.3cm}
$\bruch{K}{12} \cdot \bruch{(b-a)^3}{n^2}$
\\[0.3cm]
abgesch\"atzt werden. Hier bezeichnen $a$ und $b$ die Integrations-Grenzen, in unserem Fall
gilt also $a=0$ und $b=1$. Die Zahl $n$ gibt die Anzahl der Teil-Intervalle an und $K$ ist
eine obere Grenze f\"ur den Betrag der zweiten Ableitung der zu integrierenden Funktion in
dem Intervall $[a,b]$.  Die Aufgabe besteht also zun\"achst darin, die zweite Ableitung der
Funktion $f(x) = \exp(-x^2)$ zu berechnen und dann das Maximum zu finden, dass die zweite
Ableitung in dem Intervall $[0,1]$ annimmt.  Es gilt 
\\[0.3cm]
\hspace*{1.3cm}
$f'(x) = -2\cdot x \cdot e^{-x^2}$ \quad und \quad 
$f^{(2)}(x) = -2\cdot e^{-x^2} + 4\cdot x^2 \cdot e^{-x^2} = (4\cdot x^2 - 2)\cdot e^{-x^2}$.
\\[0.3cm]
Um das Maximum von $\bigl|f^{(2)}(x)\bigr|$ zu finden, bilden wir die Ableitung der
Funktion $f^{(2)}(x)$, wir berechnen also $f^{(3)}(x)$:
\\[0.3cm]
\hspace*{1.3cm}
$f^{(3)}(x) = \bigl(8\cdot x - 2\cdot x \cdot(4\cdot x^2 - 2)\bigr)\cdot e^{-x^2} =
   (12 \cdot x - 8 \cdot x^3)\cdot e^{-x^2}$
\\[0.3cm]
Die Funktion $f^{(2)}$ kann Extremwerte  an den Randpunkten des Intervalls $[0,1]$
annehmen und an den Stellen, wo die Ableitung dieser Funktion den Wert 0 hat.  Das f\"uhrt
auf die Gleichung 
\\[0.3cm]
\hspace*{1.3cm}
$
\begin{array}{ll}
                &  0 = (12 \cdot x - 8 \cdot x^3)\cdot e^{-x^2} \\[0.3cm]
\Leftrightarrow &  0 = (12 \cdot x - 8 \cdot x^3) \\[0.3cm]
\Leftrightarrow &  x = 0 \; \vee \; 3 - 2 \cdot x^2 = 0 \\[0.3cm]
\Leftrightarrow &  x = 0 \; \vee \; \bruch{3}{2} =  x^2  \\[0.3cm]
\Leftrightarrow &  x = 0 \; \vee \; x = \frac{1}{2}\sqrt{6} \; \vee \; x = -\frac{1}{2}\sqrt{6}
\end{array}
$
\\[0.3cm]
Da die beiden Werte $\frac{1}{2}\sqrt{6}$ und $-\frac{1}{2}\sqrt{6}$ nicht in dem
Intervall $[0,1]$ liegen, k\"onnen Extremwerte der zweiten Ableitung nur an den Stellen $0$
und $1$ auftreten.  Es gilt 
\\[0.3cm]
\hspace*{1.3cm}
$f^{(2)}(0) = -2 \cdot e^{-0^2} = -2$ \quad  und \quad
$f^{(2)}(1) = (4 -2) \cdot e^{-1} = 2/e \approx 0.7357588824$ 
\\[0.3cm]
Also ist der Betrag von $f^{(2)}(x)$ bei $x=0$ maximal und hat den Wert $2$. Damit haben
wir  $K = 2$ gefunden und die Formel f\"ur den Approximations-Fehler $e$ lautet 
\\[0.3cm]
\hspace*{1.3cm}
$ e \leq \bruch{1}{6\cdot n^2}$
\\[0.3cm]
Um das Integral mit einer Genauigkeit von $10^{-6}$ zu berechnen muss gelten 
\\[0.3cm]
\hspace*{1.3cm}
$\bruch{1}{6\cdot n^2} \leq 10^{-6} \;\Leftrightarrow\; n^2 = \bruch{10^6}{6} \;\Leftrightarrow\;
n = \bruch{10^3}{\sqrt{6}} \approx 408.2482906$
\\[0.3cm]
Als m\"ussen wir das Intervall in 409 Teil-Intervalle aufteilen um die gew\"unschte
Genauigkeit zu erreichen.
\pagebreak

\noindent
\textbf{Aufgabe}: 
\begin{enumerate}
\item Berechnen Sie mit Hilfe der Kepler'schen Fass-Regel eine Approximation 
      f\"ur das Integral 
      \\[0.1cm]
      \hspace*{1.3cm}$\displaystyle \int_0^{\frac{1}2} \sin(x)\, dx$.
\item Geben Sie eine m\"oglichst genaue Absch\"atzung f\"ur den Approximations-Fehler.
\item Vergleichen Sie ihr Ergebnis mit dem exakten Wert.
\end{enumerate}
\vspace*{0.3cm}

\noindent
\textbf{L\"osung}:  
\begin{enumerate}
\item Nach der Formel im Skript gilt
      \\[0.3cm]
      \hspace*{1.3cm}
     $\displaystyle \int_0^{\frac{1}2} \sin(x)\, dx  \approx \frac{1}{6}*\biggl( \sin(0) + 4\cdot \sin\Bigl(\frac{1}{4}\Bigr) + \cdot
        \sin\Bigl(\frac{1}{2}\Bigr) \biggr)\cdot \frac{1}{2}
        \approx 0.1224201146
       $ 
\item Nach der im Skript angegebenen Formel kann der Approximations-Fehler
      der Simpson'schen Regel durch den Ausdruck
      \\[0.3cm]
      \hspace*{1.3cm}
      $\bruch{K}{180} * \bruch{(b-a)^5}{n^4}$
      \\[0.3cm]
      abgesch\"atzt werden.  Hier ist $K$  eine Absch\"atzung f\"ur die vierte Ableitung des
      Integranden und $a$ und $b$ sind die Intervall-Grenzen, in unserem Fall gilt also
      $a=0$ und $b=\frac{1}{2}$.  F\"ur die Kepler'sche Fass-Regel gilt au\3erdem $n=2$.
      Wir berechnen nun die Ableitungen der Funktion $f(x) = \sin(x)$: 
      \\[0.3cm]
      \hspace*{1.3cm}
      $f'(x) = \cos(x)$, $f^{(2)}(x) = -\sin(x)$, $f^{(3)}(x) = -\cos(x)$ und $f^{(4)}(x) = \sin(x)$.
      \\[0.3cm]
      Die Ableitung von $x \mapsto \sin(x)$ ist $x \mapsto \cos(x)$.  Diese Funktion
      hat in dem  Intervall $\bigl[0, \frac{1}{2}\bigr]$ keine Nullstelle. Also nimmt die Funktion
      $\sin(x)$ die Extremwerte an den Randpunkten an.  Es gilt $\sin(0) = 0$ und
      $\sin\bigl(\frac{1}{2}\bigr) = 0.4794255386$ und damit gilt 
      \\[0.3cm]
      \hspace*{1.3cm}
      $K = \sin\bigl(\frac{1}{2}\bigr)$.
      \\[0.3cm]
      Wegen $n=2$ gilt f\"ur den Approximations-Fehler $e$ also
      \\[0.1cm]
      \hspace*{1.3cm}
      $e \leq \frac{1}{180}\cdot \sin\bigl(\frac{1}{2}\bigr)\cdot \frac{1}{2^4} \cdot \bigl(\frac{1}{2}\bigr)^5 = \frac{1}{92160} \sin\bigl(\frac{1}{2}\bigr) 
       \approx 5.202100026 \cdot 10^{-6}$
\item F\"ur den exakten Wert gilt
      \\[0.3cm]
      \hspace*{1.3cm}
      $\displaystyle \int_0^{\frac{1}2} \sin(x)\, dx = -\cos(x) \bigg|_0^{\frac{1}{2}} = \cos(0) - \cos\bigl(\frac{1}{2}\bigr) \approx 0.1224174381$
      \\[0.3cm]
      Der tats\"achliche Fehler betr\"agt $0.1224174381 - 0.1224201146 \approx 2.6765\cdot 10^{-6}$.
\end{enumerate}
\pagebreak

\noindent
\textbf{Aufgabe}: Gegenstand dieser Aufgabe ist die numerische Berechnung der Summe 
\\[0.1cm]
\hspace*{1.3cm} $\displaystyle \sum\limits_{k=1}^\infty \bruch{1}{k^3}$.
\\[0.1cm]
Gehen Sie zur Berechnung dieser Summe in folgenden Schritten vor.
\begin{enumerate}
\item Approximieren Sie die Rest-Summe $\sum\limits_{k=n}^\infty \bruch{1}{k^3}$ 
      durch ein geeignetes Integral.

      \textbf{Hinweis}: Es gilt 
      \\[0.1cm]
      \hspace*{1.3cm} $\displaystyle f(k) = \int_{k-\frac{1}{2}}^{k+\frac{1}{2}} f(k) \, dt \approx \int_{k-\frac{1}{2}}^{k+\frac{1}{2}} f(t) \, dt$.
\item Berechnen Sie eine Absch\"atzung f\"ur den Approximations-Fehler,
      den Sie bei der Integration in Teil (a) erhalten.
      
      \textbf{Hinweis}: Approximieren Sie die auftretenden Summen durch Integrale.
\item Berechnen Sie nun, wir gro\3 Sie $n$ w\"ahlen m\"ussen, damit der Approximations-Fehler
      kleiner als $10^{-6}$ bleibt.
\item Geben Sie nun einen N\"aherungs-Wert f\"ur die Summe $\sum\limits_{k=1}^\infty \bruch{1}{k^3}$,
      der sich von dem exakten Ergebnis um weniger als $10^{-6}$ unterscheidet.
\end{enumerate}
\vspace*{0.3cm} 

\noindent
\textbf{L\"osung}: 
\begin{enumerate}
\item Dem Hinweis folgend approximieren wir $f(k)$ durch 
      \\[0.1cm]
      \hspace*{1.3cm} $\displaystyle f(k) \approx \int_{k-\frac{1}{2}}^{k+\frac{1}{2}} f(t) \, dt$.
      \\[0.1cm]
      Damit finden wir f\"ur die Rest-Summe die N\"aherungs-Formel
      \\[0.1cm]
      \hspace*{1.3cm}
      $\displaystyle
       \sum\limits_{k=n}^\infty \bruch{1}{k^3} \approx 
       \sum\limits_{k=n}^\infty \int_{k-\frac{1}{2}}^{k+\frac{1}{2}} \bruch{1}{t^3} \, dt =
       \int_{n-\frac{1}{2}}^{\infty} \bruch{1}{t^3} \, dt =
       -\frac{1}{2}\cdot\bruch{1}{t^2} \bigg|_{n-\frac{1}{2}}^{\infty} =
       \frac{1}{2} \cdot \bruch{1}{\Bigl(n-\frac{1}{2}\Bigr)^2} 
       $
\item Der Approximations-Fehler $e(k)$, den wir bei der N\"aherung von $\frac{1}{k^3}$ machen, hat
      den Wert 
      \\[0.1cm]
      \hspace*{1.3cm}
      $
      \begin{array}[t]{lcl}
      e(k) & = &\bruch{1}{k^3} - \int_{k-\frac{1}{2}}^{k+\frac{1}{2}} \bruch{1}{t^3}\, dt \\[0.5cm]
           & = & \bruch{1}{k^3} + \frac{1}{2}\left(\bruch{1}{t^2} \bigg|_{k-\frac{1}{2}}^{k+\frac{1}{2}}\right) \\[0.5cm]
           & = & \bruch{1}{k^3} + \frac{1}{2}\cdot \Biggl(\frac{1}{\bigl(k+\frac{1}{2}\bigr)^2} - \frac{1}{\bigl(k-\frac{1}{2}\bigr)^2}\Biggr)  \\[0.5cm]
      \end{array}
      $
      \\[0.1cm]
      Den gesamten Approximations-Fehler erhalten wir, wenn wir diesen Fehler f\"ur $n=k$
      bis $\infty$ aufsummieren.  Die dabei auftretenden Summen approximieren wir \"ahnlich
      wie in Teil (a) der Aufgabe durch Integrale.
      \\[0.1cm]
      \hspace*{1.3cm}
      $
      \begin{array}[t]{lcl}
         e & = & \displaystyle 
                 \sum\limits_{k=n}^\infty
       \bruch{1}{k^3} + \frac{1}{2}\cdot \Biggl(\frac{1}{\bigl(k+\frac{1}{2}\bigr)^2} - \frac{1}{\bigl(k-\frac{1}{2}\bigr)^2}\Biggr)  \\[0.7cm]
           & \approx & \displaystyle 
                 \int_{n-\frac{1}{2}}^\infty
       \bruch{1}{t^3} + \frac{1}{2}\cdot \Biggl(\frac{1}{\bigl(t+\frac{1}{2}\bigr)^2} - \frac{1}{\bigl(t-\frac{1}{2}\bigr)^2}\Biggr)\, dt  \\[0.5cm]
      \end{array}
      $
      \\[0.1cm]
      Also haben wir
      \\[0.1cm]
      \hspace*{1.3cm}
      $
      \begin{array}[t]{lcl}
         e & \approx & \displaystyle 
       -\frac{1}{2}\cdot\bruch{1}{t^2} - \frac{1}{2}\cdot \biggl(\frac{1}{t+\frac{1}{2}} - \frac{1}{t-\frac{1}{2}}\biggr)\biggl|_{n-\frac{1}{2}}^\infty  \\[0.5cm]
           & = & \displaystyle 
           \frac{1}{2}\cdot \biggl( \frac{1}{\bigl(n-\frac{1}{2}\bigr)^2} + \frac{1}{n} - \frac{1}{n-1}\biggr) \\[0.5cm]
           & = & \displaystyle 
           \frac{1}{2}\cdot \biggl( \frac{1}{\bigl(n-\frac{1}{2}\bigr)^2} + \frac{n-1 - n}{n\cdot(n-1)}\biggr) \\[0.5cm]
           & = & \displaystyle 
           \frac{1}{2}\cdot \biggl( \frac{1}{\bigl(n-\frac{1}{2}\bigr)^2} - \frac{1}{n\cdot(n-1)}\biggr) \\[0.5cm]
           & = & \displaystyle 
           \frac{1}{2}\cdot \frac{n^2-n - \bigl(n^2 - n + \frac{1}{4}\bigr)}{\bigl(n-\frac{1}{2}\bigr)^2 \cdot n \cdot (n-1)} \\[0.5cm]
           & = & \displaystyle 
           -\frac{1}{8}\cdot \frac{1}{\bigl(n-\frac{1}{2}\bigr)^2 \cdot n \cdot (n-1)} \\[0.5cm]
           & \approx & \displaystyle 
           -\frac{1}{8}\cdot \frac{1}{n^4} 
      \end{array}
      $
\itemwww. Um die Summe $\sum\limits_{k=1}^\infty \bruch{1}{k^3}$ mit einer Genauigkeit von
      $10^{6}$ zu berechnen m\"ussen wir $n$ so w\"ahlen, dass gilt 
      \\
      \hspace*{1.3cm}
      $\frac{1}{8}\cdot \bruch{1}{n^4} \leq 10^{-6} \;\Leftrightarrow\; n \geq
      \sqrt[4]{\bruch{10^6}{8}} \approx 18.80301547
      $.
      \\[0.1cm]
      Also m\"ussen wir $n = 19$ w\"ahlen um die geforderte Genauigkeit zu erhalten.
\item Wir haben 
      \\[0.1cm]
      \hspace*{1.3cm}
      $\sum\limits_{k=1}^\infty \bruch{1}{k^3} \approx \sum\limits_{k=1}^{n-1}
      \bruch{1}{k^3} + \frac{1}{2} \cdot \frac{1}{\bigl(n-\frac{1}{2}\bigr)^2} \approx 1.202057968
      $
      %maple evalf(sum(1/k^3, k=1.. n-1) + 1/2 * 1/(n-1/2)^2 - 1/8 * 1/(n*(n-1/2)^2*(n-1)))
      \\[0.1cm]
      Der Fehler des so berechneten Werts betr\"agt $1.065 \cdot 10^{-6}$.
\end{enumerate}

\pagebreak

\noindent
\textbf{Aufgabe}:
Die Funktion $p$ sei auf dem Intervall $[-\pi,\pi]$ definiert durch
\\[0.1cm]
\hspace*{1.3cm}
$p(x) = x^2$.
\\[0.1cm]
Die Funktion werde so auf $\mathbb{R}$ fortgesetzt, dass die resultierende Funktion die Periode
$2\!\cdot\!\pi$ hat.  
\begin{enumerate}
\item Berechnen Sie die Fourier-Reihe von $p$.
\item Berechnen Sie mit Hilfe der Fourier-Reihe von $p$ einen Wert f\"ur die Summe
      \\[0.1cm]
      \hspace*{1.3cm}
      $\displaystyle \sum\limits_{n=1}^\infty \bruch{1}{n^2}$. 
\end{enumerate}

\noindent
\textbf{L\"osung}: Wir berechnen zun\"achst die Fourier-Reihe.  Da die Funktion $p$ gerade
ist, haben die Koeffizienten $b_k$ alle den Wert $0$.  F\"ur den Koeffizienten $a_0$ gilt 
\\[0.1cm]
\hspace*{1.3cm}
$\displaystyle a_0 = 
\bruch{2}{\pi}\cdot\int_0^{\pi} p(x)\,dx = 
\bruch{2}{\pi}\cdot\int_0^{\pi} x^2\,dx = 
\bruch{2}{\pi}\cdot \bruch{x^3}{3} \bigg|_0^{\pi} = 
\bruch{2}{\pi}\cdot \bruch{\pi^3}{3}  = 
\bruch{2}{3}\cdot\pi^2$. 
\\[0.1cm]
F\"ur $k>0$ berechnen wir die Koeffizienten $a_k$ durch zweimalige partielle Integration.
\\[0.3cm]
\hspace*{1.3cm}
$
\begin{array}[t]{lcl} 
\bruch{\pi}{2}\cdot a_k 
& = & \displaystyle
\int_0^{\pi}\!\! x^2\cdot \cos(n\!\cdot\!x)\, dx \\[0.5cm]
& = & \displaystyle
x^2\cdot\bruch{1}{n}\cdot\sin(n\!\cdot\!x) \,\bigg|_0^{\pi} - \bruch{2}{n}\cdot \int_0^{\pi}\!\! x\cdot \sin(n\!\cdot\!x)\, dx \\[0.5cm]
& = & \displaystyle
-\bruch{2}{n}\left( x\cdot\bruch{-1}{n}\cdot\cos(n\!\cdot\!x) \,\bigg|_0^{\pi} + \bruch{1}{n}\cdot \int_0^{\pi}\!\! \cos(n\!\cdot\!x)\, dx \right)\\[0.5cm]
& = & \displaystyle
-\bruch{2}{n}\left( \pi\cdot\bruch{(-1)^{n+1}}{n} + \bruch{1}{n^2}\cdot \sin(n\!\cdot\!x)\bigg|_0^{\pi} \right)\\[0.5cm]
& = & \displaystyle
\bruch{2\cdot\pi}{n^2} \cdot (-1)^n 
\end{array}
$
\\[0.3cm]
Damit haben f\"ur $a_k$ den Wert 
\\[0.1cm]
\hspace*{4.3cm}
$a_k = 4\cdot \bruch{(-1)^n}{n^2}$
\\[0.1cm]
gefunden.  Also gilt 
\\[0.1cm]
\hspace*{4.3cm}
$p(x) = \bruch{1}{3} \pi^2 + 4\cdot\sum\limits_{n=1}^\infty \bruch{(-1)^n}{n^2}\cdot\cos(n\!\cdot\!x)$
\\[0.3cm]
Setzen wir in dieser Gleichung f\"ur $x$ den Wert $\pi$ ein, so  folgt wegen 
$p(\pi) = \pi^2$ und $\cos(n\!\cdot\!\pi) = (-1)^n$ die Gleichung:
\\[0.3cm]
\hspace*{1.3cm}
$
\begin{array}[t]{llcl}
               & \pi^2  & = & \bruch{1}{3}\cdot \pi^2 + 4 \cdot \sum\limits_{n=1}^\infty \bruch{1}{n^2} \\[0.5cm]
\Leftrightarrow& \bruch{2}{3} \cdot \pi^2  & = &  4 \cdot\sum\limits_{n=1}^\infty \bruch{1}{n^2} \\[0.5cm]
\Leftrightarrow& \bruch{1}{6} \cdot \pi^2  & = &  \sum\limits_{n=1}^\infty \bruch{1}{n^2} \\[0.5cm]
\end{array}
$

\pagebreak
\noindent
\textbf{Aufgabe}:  Die Lambert'sche W-Funktion (Johann Heinrich Lambert; 1728 - 1777)
$x \mapsto W(x)$ ist f\"ur $x\geq 0$ definiert als die Umkehr-Funktion der Funktion $x \mapsto x\cdot e^x$,
es gilt also 
\\[0.1cm]
\hspace*{1.3cm} $\displaystyle W(x) \cdot e^{W(x)} = x$ \quad f\"ur alle $x \in \mathbb{R}_+$.
\begin{enumerate}
\item Berechnen Sie die Ableitung der Lambert'schen W-Funktion.
\item Formen Sie den Ausdruck f\"ur $W'(x)$ so um, dass der Term $e^{W(x)}$ nicht mehr
      auftritt.
\item Berechnen Sie eine Stamm-Funktion der Lambert'schen W-Funktion.
\item Nehmen Sie an, dass Sie die Lambert'sche W-Funktion berechnen k\"onnen und
      bestimmen Sie unter dieser Annahme f\"ur ein gegebenes $\varepsilon$ die L\"osung der Gleichung 
      \\[0.1cm]
      \hspace*{1.3cm}
      $\bruch{1}{n} * \Bigl(\frac{1}{2}\Bigr)^n = \varepsilon$
      \\[0.1cm]
      durch algebraische Umformungen.

      \noindent
      \textbf{Hinweis}: Invertieren Sie die Gleichung und bringen Sie die Gleichung dann
      auf die Form $\alpha * e^\alpha = \beta$ f\"ur geeignete $\alpha$ und $\beta$, denn
      dann gilt $\alpha = W(\beta)$.
\end{enumerate}
\vspace*{0.3cm}

\noindent
\textbf{L\"osung}:
\begin{enumerate}
\item Um die Ableitung der Lambert'schen W-Funktion zu bestimmen, leiten wir die Gleichung
      \\[0.1cm]
      \hspace*{1.3cm}
       $\displaystyle W(x) \cdot e^{W(x)} = x$ 
      \\[0.1cm]
      nach $x$ ab.  Das liefert: 
      \\[0.3cm]
      \hspace*{0.0cm}
      $
      \begin{array}[t]{lrcll}
        & \displaystyle 
        W'(x) \cdot e^{W(x)} + W(x) \cdot W'(x) \cdot e^{W(x)} & = & 1 \\[0.1cm]
      \Leftrightarrow & \displaystyle 
        W'(x) \cdot \bigl(1 + W(x)\bigr) \cdot e^{W(x)} & = & 1 & \hspace*{-1cm}
                      \bigg|\; \cdot \bruch{1}{\bigl(1+W(x)\bigr) \cdot e^{W(x)}} \\[0.1cm]
        
      \Leftrightarrow & \displaystyle 
        W'(x)  & = & \bruch{1}{\bigl(1+W(x)\bigr) \cdot e^{W(x)}} \\[0.1cm]
      \end{array}
      $
\item Nach der Definition der Lambert'schen W-Funktion gilt 
      \\[0.3cm]
      \hspace*{1.3cm} 
       $\displaystyle W(x) \cdot e^{W(x)} = x$ \quad also \quad $e^{W(x)} = \bruch{x}{W(x)}$.
      \\[0.3cm]
      Setzen wir dies in die obige Gleichung f\"ur $W'(x)$ ein, so erhalten wir 
      \\[0.1cm]
      \hspace*{1.3cm}
      $W'(x) = \bruch{W(x)}{x*\bigl(1 + W(x)\bigr)}$.
\item Um die Formel zur Berechnung des unbestimmten Integrals einer Umkehr-Funktion aus
      Abschnitt 4.2.3 des Skripts verwenden zu k\"onnen, berechnen wir zun\"achst das
      unbestimmte Integral der Funktion $x \mapsto x\cdot e^x$ durch partielle Integration:
      \\[0.1cm]
      \hspace*{1.3cm}
      $\displaystyle G(x) := \int x\cdot e^x\, dx = x\cdot e^x - \int e^x\, dx = x\cdot e^x - e^x = (x -1)\cdot e^x$
      \\[0.1cm]
      Nun gilt 
      \\[0.1cm]
      \hspace*{1.3cm}
      $
      \begin{array}[t]{lcl}
      \displaystyle \int W(x)\, dx & = & x\cdot W(x) - G\bigl(W(x)\bigr) \\[0.1cm]
                                   & = & x\cdot W(x) - \bigl(W(x) - 1\bigr)\cdot e^{W(x)} \\[0.1cm]
                                   & = & x\cdot W(x) - \bigl(W(x) - 1\bigr)\cdot \bruch{x}{W(x)} \\[0.1cm]
                                   & = & \bruch{W^2(x) - W(x) + 1}{W(x)} \cdot x. \\[0.1cm]
      \end{array}
      $
\item Es gilt: 
      \\[0.1cm]
      \hspace*{1.3cm}
      $
      \begin{array}[t]{lrcll}
        & \bruch{1}{n} \cdot \Bigl(\frac{1}{2}\Bigr)^n & = & \varepsilon \\[0.3cm]
      \Leftrightarrow & \displaystyle n \cdot 2^n & = & \bruch{1}{\varepsilon} \\[0.3cm]
      \Leftrightarrow & \displaystyle n \cdot \bigl(e^{\ln(2)}\bigr)^n & = & \bruch{1}{\varepsilon} & \\[0.3cm]
      \Leftrightarrow & \displaystyle \bigl(\ln(2)\cdot n\bigr) \cdot e^{\ln(2)\cdot n} & = & \bruch{\ln(2)}{\varepsilon} \\[0.3cm]
      \Leftrightarrow & \displaystyle \ln(2)\cdot n & = & W\Bigl(\bruch{\ln(2)}{\varepsilon}\Bigr) & \\[0.3cm]
      \Leftrightarrow & \displaystyle n & = & \bruch{1}{\ln(2)} \cdot W\Bigl(\bruch{\ln(2)}{\varepsilon}\Bigr) & 
      \end{array}
      $
      \\[0.1cm]
\end{enumerate}
\pagebreak

\renewcommand{\labelenumi}{\arabic{enumi}.}
\begin{enumerate}
\item Zeigen Sie anhand der Definition des Grenzwerts, dass 
      $\lim\limits_{n \rightarrow \infty} \bruch{1}{\sqrt{n}} = 0$ ist.
%\item Zeigen Sie, dass die Folge $\folge{\sin(n)}$ nicht konvergent ist.
%      leider zu schwer
\item Es seien $a,b\in\mathbb{R}$ gegeben. 
      Die Folge $\folge{c_n}$ werde induktiv definiert durch $c_0 = a$, $c_1 = b$ und  
      \\[0.1cm]
      \hspace*{1.3cm} $c_{n+2} = c_n + c_{n+1}$.
      \begin{enumerate}
      \item Zeigen Sie durch Induktion nach $n$, dass gilt: 
            \\[0.1cm]
            \hspace*{1.3cm}
            $c_n = \bruch{1}{3}\cdot a \biggl(1-\Bigl(\frac{-1}{2}\Bigr)^{n-1}\biggr) + \bruch{2}{3}\cdot b \biggl(1-\Bigl(\frac{-1}{2}\Bigr)^{n}\biggr)$
      \item Berechnen Sie den Grenzwert $\lim\limits_{n \rightarrow \infty} c_n$.
      \end{enumerate}
\end{enumerate}
\pagebreak

\noindent
\textbf{Aufgabe}: Gegenstand dieser Aufgabe ist die numerische Berechnung der Summe 
\\[0.1cm]
\hspace*{1.3cm} $\displaystyle \sum\limits_{k=1}^\infty \bruch{1}{k^2}$.
\\[0.1cm]
Gehen Sie zur Berechnung dieser Summe in folgenden Schritten vor.
\begin{enumerate}
\item Approximieren Sie die Rest-Summe $\sum\limits_{k=n}^\infty \bruch{1}{k^2}$ 
      durch ein geeignetes Integral. 
      \hspace*{\fill} (8 Punkte)

      \textbf{Hinweis}: Es gilt 
      \\[0.1cm]
      \hspace*{1.3cm} $\displaystyle f(k) = \int_{k-\frac{1}{2}}^{k+\frac{1}{2}} f(k) \, dt \approx \int_{k-\frac{1}{2}}^{k+\frac{1}{2}} f(t) \, dt$.
\item Berechnen Sie eine Absch\"atzung f\"ur den Approximations-Fehler,
      den Sie bei der Integration in Teil (a) erhalten.
      
      \textbf{Hinweis}: Approximieren Sie die auftretenden Summen durch Integrale.
\item Berechnen Sie nun, wir gro\3 Sie $n$ w\"ahlen m\"ussen, damit der Approximations-Fehler
      kleiner als $10^{-6}$ bleibt.
\end{enumerate}
\vspace*{0.3cm} 

\noindent
\textbf{L\"osung}: 
\begin{enumerate}
\item Dem Hinweis folgend approximieren wir $f(k)$ durch 
      \\[0.1cm]
      \hspace*{1.3cm} $\displaystyle f(k) \approx \int_{k-\frac{1}{2}}^{k+\frac{1}{2}} f(t) \, dt$.
      \\[0.1cm]
      Damit finden wir f\"ur die Rest-Summe die N\"aherungs-Formel
      \\[0.1cm]
      \hspace*{1.3cm}
      $\displaystyle
       \sum\limits_{k=n}^\infty \bruch{1}{k^2} \approx 
       \sum\limits_{k=n}^\infty \int_{k-\frac{1}{2}}^{k+\frac{1}{2}} \bruch{1}{t^2} \, dt =
       \int_{n-\frac{1}{2}}^{\infty} \bruch{1}{t^2} \, dt =
       -\bruch{1}{t} \bigg|_{n-\frac{1}{2}}^{\infty} =
       \bruch{1}{n-\frac{1}{2}} 
       $
\item Der Approximations-Fehler $e(k)$, den wir bei der N\"aherung von $\frac{1}{k^2}$ machen, hat
      den Wert 
      \\[0.1cm]
      \hspace*{1.3cm}
      $
      \begin{array}[t]{lcl}
      e(k) & = &\bruch{1}{k^2} - \int_{k-\frac{1}{2}}^{k+\frac{1}{2}} \bruch{1}{t^2}\, dt \\[0.5cm]
           & = & \bruch{1}{k^2} + \left(\bruch{1}{t} \bigg|_{k-\frac{1}{2}}^{k+\frac{1}{2}}\right) \\[0.5cm]
           & = & \bruch{1}{k^2} + \Biggl(\frac{1}{k+\frac{1}{2}} - \frac{1}{k-\frac{1}{2}}\Biggr)  \\[0.5cm]
      \end{array}
      $
      \\[0.1cm]
      Den gesamten Approximations-Fehler erhalten wir, wenn wir diesen Fehler f\"ur $n=k$
      bis $\infty$ aufsummieren.  Die dabei auftretenden Summen approximieren wir \"ahnlich
      wie in Teil (a) der Aufgabe durch Integrale.
      \\[0.1cm]
      \hspace*{1.3cm}
      $
      \begin{array}[t]{lcl}
         e & = & \displaystyle 
                 \sum\limits_{k=n}^\infty
       \bruch{1}{k^2} + \Biggl(\frac{1}{k+\frac{1}{2}} - \frac{1}{k-\frac{1}{2}}\Biggr)  \\[0.7cm]
           & \approx & \displaystyle 
                 \int_{n-\frac{1}{2}}^\infty
       \bruch{1}{t^2} + \Biggl(\frac{1}{t+\frac{1}{2}} - \frac{1}{t-\frac{1}{2}}\Biggr)\, dt  \\[0.5cm]
      \end{array}
      $
      \\[0.1cm]
      Also haben wir
      \\[0.1cm]
      \hspace*{1.3cm}
      $
      \begin{array}[t]{lcl}
         e & \approx & \displaystyle 
       -\bruch{1}{t} + \Bigl(\ln\bigl(t+\frac{1}{2}\bigr) - \ln\bigl(t-\frac{1}{2}\bigr)\Bigr)\biggl|_{n-\frac{1}{2}}^\infty  \\[0.5cm]
           & = & \displaystyle 
       \bruch{1}{n-\frac{1}{2}} + \ln(n) - \ln(n-1)  \\[0.5cm]
           & = & \displaystyle 
       \bruch{1}{n-\frac{1}{2}}  - \ln\Bigl(1-\frac{1}{n}\Bigr)  \\[0.5cm]
      \end{array}
      $
\end{enumerate}
\end{document}





%%% Local Variables: 
%%% mode: latex
%%% TeX-master: t
%%% End: 
