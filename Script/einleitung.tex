\chapter{Einleitung}
Der vorliegende Text ist das Skript zu meiner Analysis-Vorlesung f�r Informatiker.  Ich habe mich
bei der Ausarbeitung dieser Vorlesung im wesentlichen auf die folgenden Lehrb�cher gest�tzt:
\begin{enumerate}
\item \emph{Analysis} \texttt{I} von Otto Forster \cite{forster:2011}.
\item \emph{Differential- und Integralrechnung \texttt{I}} von Hans Grauert und Ingo Lieb \cite{grauert:1967}.
\item \emph{Differential and Integral Calculus, Volume 1} von Richard Courant  \cite{courant:1937}.
\item \emph{Advanced Calculus} von Richard Wrede und Murray R.~Spiegel \cite{wrede:2010}.
\end{enumerate}
Den Studenten empfehle ich das erste Buch in dieser Liste, denn dieses Buch ist auch in
elektronischer Form in unserer Bibliothek vorhanden.  Bei dem Buch von Richard Courant ist das
Copyright mittlerweile abgelaufen, so dass Sie es im Netz unter
\\[0.2cm]
\hspace*{0.3cm}
\href{https://ia700700.us.archive.org/34/items/DifferentialIntegralCalculusVolI/Courant-DifferentialIntegralCalculusVolI.pdf}{\texttt{https://ia700700.us.archive.org/}
\\
\hspace*{0.3cm}
\texttt{34/items/DifferentialIntegralCalculusVolI/Courant-DifferentialIntegralCalculusVolI.pdf}}
\\[0.2cm]
finden k�nnen.  Schlie�lich enth�lt das Buch von Wrede und Spiegel eine Vielzahl gel�ster
Aufgaben und bietet sich daher besonders zum �ben an.

\section{�berblick �ber die Vorlesung}
Im Rahmen der Vorlesung werden die folgenden Gebiete behandelt:
\begin{enumerate}
\item Da sich die Analysis mit den reellen Zahlen besch�ftigt, beginnen wir damit, dass wir die
      Menge $\mathbb{R}$ der reellen Zahlen zun�chst axiomatisch als vollst�ndig geordneten K�rper
      charakterisieren und dann die Menge $\mathbb{R}$ mit Hilfe von   
      \href{http://de.wikipedia.org/wiki/Dedekindscher_Schnitt}{\emph{Dedekindschen-Schnitten}}
      definieren.
\item Darauf aufbauend  f�hren den f�r den Rest der Vorlesung grundlegenden  Begriffe des 
      \href{http://de.wikipedia.org/wiki/Grenzwert_(Folge)}{\emph{Grenzwerts}} einer
      \href{http://de.wikipedia.org/wiki/Folge_(Mathematik)}{\emph{Folge}} ein. 
      
      Wir werden beispielsweise sehen, dass die Folge $(b_1,b_2, b_3, \cdots)$, die induktiv durch
      \\[0.2cm]
      \hspace*{1.3cm}
      $b_1 := 1$ \quad und \quad $\ds b_{n+1} := \frac{1}{2} \cdot \left(b_n + \frac{2}{b_n}\right)$ f�r alle $n \in \mathbb{N}$
      \\[0.2cm]
      definiert ist, gegen $\sqrt{2}$ \emph{konvergiert}.
\item Das vierte Kapitel diskutiert die Begriffe 
      \href{http://de.wikipedia.org/wiki/Stetigkeit}{\emph{Stetigkeit}} und 
      \href{http://de.wikipedia.org/wiki/Differenzierbarkeit}{\emph{Differenzierbarkeit}}.
\item Das f�nfte Kapitel zeigt verschiedene Anwendungen der bis dahin dargestellten Theorie.
      Insbesondere werden 
      \href{http://de.wikipedia.org/wiki/Taylorreihe}{\emph{Taylor-Reihen}} diskutiert. Diese k�nnen beispielsweise zur
      Berechnung der trigonometrischen Funktionen verwendet werden.  So werden wir beispielsweise
      sehen, dass wir f�r eine reelle Zahl $x$ den Wert $\sin(x)$ durch den Ausdruck
      \\[0.2cm]
      \hspace*{1.3cm}
      $\ds \sum\limits_{n=0}^{\infty} (-1)^n \cdot \frac{x^{2 \cdot n + 1}}{(2 \cdot n + 1)!} = x - \frac{1}{6} \cdot x^3 + \frac{1}{120} \cdot x^5 \mp \cdots$
      \\[0.2cm]
      berechnen k�nnen.  Au�erdem diskutieren wir in diesem Kapitel Verfahren zur numerischen L�sung
      von Gleichungen.  Beispielsweise zeigen wir, wie wir die Gleichung
      \\[0.2cm]
      \hspace*{1.3cm}
      $\cos(x) = x$ 
      \\[0.2cm]
      numerisch mit Hilfe einer Fixpunkt-Iteration l�sen k�nnen.
\item Das sechste Kapitel besch�ftigt sich mit der
      \href{http://de.wikipedia.org/wiki/Integralrechnung}{\emph{Integralrechnung}}. 
\item Im siebten Kapitel zeigen wir, dass die Kreiszahl  \href{http://de.wikipedia.org/wiki/Kreiszahl}{$\pi$} 
      und die eulersche Zahl \href{http://de.wikipedia.org/wiki/Eulersche_Zahl}{$e$} keine rationalen Zahlen sind.
\item Im achten Kapitel diskutieren wir \href{http://de.wikipedia.org/wiki/Fourierreihe}{\emph{Fourier-Reihen}}.
      Unter anderem werden wir zeigen, dass
      \\[0.2cm]
      \hspace*{1.3cm}
      $\ds \sum\limits_{n=1}^{\infty} \frac{1}{n^2} = \frac{\pi^2}{6}$
      \\[0.2cm] 
      gilt.  Fourierreihen sind eine der Grundlage der Theorie der
      \href{http://de.wikipedia.org/wiki/Digitale_Signalverarbeitung}{\emph{digitalen Signalverarbeitung}}.
\item Das letzte Kapitel gibt einen kurzen Ausblick auf
      \href{http://de.wikipedia.org/wiki/Rundungsfehler}{\emph{Rundungs-Fehler}}. 
\end{enumerate}
Aus Zeitgr�nden werden wir im Laufe der Vorlesung das Skript nicht vollst�ndig abdecken k�nnen.
Daher wird der Stoff des siebten Kapitels im Rahmen der Vorlesung vermutlich nur oberfl�chlich
behandelt werden. 

\section{Ziel der Vorlesung}
Wir werden im Rahmen der Vorlesung nicht die Zeit haben, alle Aspekte der Analysis zu besprechen.
Insbesondere werden wir viele interessante Anwendungen der Analysis in der Informatik nicht
diskutieren k�nnen.  Das ist aber auch gar nicht das Ziel dieser Vorlesung:  Mir geht es vor allem
darum, Ihnen die F�higkeit zu vermitteln, sich 
selbstst�ndig in mathematische Fachliteratur einarbeiten zu k�nnen.  Dazu m�ssen Sie in der Lage
sein, mathematische Beweise sowohl zu verstehen als auch selber entwickeln zu k�nnen.   Dies ist ein
wesentlicher Unterschied zu der Mathematik, an die sich viele von Ihnen auf der Schule gew�hnt haben:
Dort werden prim�r Verfahren vermitteln, mit denen sich spezielle Probleme l�sen lassen.  Die Kenntnis
solcher Verfahren ist allerdings in der Praxis nicht mehr so wichtig wie fr�her, denn heutzutage
werden solche Verfahren als Programme implementiert und daher besteht kein Bedarf mehr daf�r, diese Verfahren von Hand
anzuwenden.  Daher wird in dieser Vorlesung der mathematische Beweis-Begriff im
Vordergrund stehen.  Die Analysis dient uns dabei als ein Beispiel einer mathematischen Theorie, an
Hand derer wir das mathematische Denken �ben k�nnen. 

\section{Notation}
In diesem Skript definieren wir die Menge der nat�rlichen Zahlen $\mathbb{N}$ �ber die Formel
\\[0.2cm]
\hspace*{1.3cm}
$\mathbb{N} := \{ 1, 2, 3, \cdots \}$.
\\[0.2cm]
Weiter definieren wir
\\[0.2cm]
\hspace*{1.3cm}
$\mathbb{N}_0 := \{ 0 \} \cup \mathbb{N}$.
\\[0.2cm]
Im �brigen schlie�t dieses Skript an mein
\href{https://github.com/karlstroetmann/Lineare-Algebra/blob/master/Script/lineare-algebra.pdf}{Skript zur linearen Algebra} 
an und verwendet die selben Notationen.
\pagebreak

\section{Eine Bitte}
Dieses Skript enth�lt sicher noch den einen oder anderen Fehler.  Sollte Ihnen ein Fehler auffallen, so bitte ich
um einen Hinweis unter der Adresse
\\[0.2cm]
\hspace*{1.3cm}
\href{mailto:karl.stroetmann@dhbw-mannheim.de}{karl.stroetmann@dhbw-mannheim.de}.
\\[0.2cm]
Es bringt mir wenig, wenn Sie mich innerhalb meiner Vorlesung auf einen Tipp- oder
Rechtschreibfehler hinweisen, denn bis ich dazu komme, einen solchen Fehler zu korrigieren, habe ich
ihn meistens schon wieder vergessen.  Daher die Bitte mir solche Fehler per Email mitzuteilen.
Wenn Sie mit \href{http://github.com}{\texttt{github}} vertraut sind, k�nnen Sie mir auch gerne einen
\href{https://help.github.com/articles/using-pull-requests}{\textsl{Pull Request}} schicken.


%%% Local Variables: 
%%% mode: latex
%%% TeX-master: "analysis"
%%% End: 
