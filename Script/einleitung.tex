\chapter{Einleitung}
Der vorliegende Text ist das Skript zu meiner Analysis-Vorlesung f\"ur Informatiker.  Ich habe mich
bei der Ausarbeitung dieser Vorlesung im wesentlichen auf die folgenden Lehrb\"ucher gest\"utzt:
\begin{enumerate}
\item \emph{Analysis} \texttt{I} von Otto Forster \cite{forster:2011}.
\item \emph{Differential- und Integralrechnung \texttt{I}} von Hans Grauert und Ingo Lieb \cite{grauert:1967}.
\item \emph{Differential and Integral Calculus, Volume 1} von Richard Courant  \cite{courant:1937}.
\item \emph{Advanced Calculus} von Richard Wrede und Murray R.~Spiegel \cite{wrede:2010}.
\end{enumerate}
Den Studenten empfehle ich das erste Buch in dieser Liste, denn dieses Buch ist auch in
elektronischer Form in unserer Bibliothek vorhanden.  Bei dem Buch von Richard Courant ist das
Copyright mittlerweile abgelaufen, so dass Sie es im Netz unter
\\[0.2cm]
\hspace*{0.3cm}
\href{https://archive.org/details/DifferentialIntegralCalculusVolI}{\texttt{https://archive.org/details/DifferentialIntegralCalculusVolI}}
\\[0.2cm]
finden k\"onnen.  Schlie\ss{}lich enth\"alt das Buch von Wrede und Spiegel eine Vielzahl gel\"oster
Aufgaben und bietet sich daher besonders zum \"uben an.

\section{\"Uberblick \"uber die Vorlesung}
Im Rahmen der Vorlesung werden die folgenden Gebiete behandelt:
\begin{enumerate}
\item Da sich die Analysis mit den reellen Zahlen besch\"aftigt, beginnen wir damit, dass wir die
      Menge $\mathbb{R}$ der reellen Zahlen zun\"achst axiomatisch als vollst\"andig geordneten K\"orper
      charakterisieren und dann die Menge $\mathbb{R}$ mit Hilfe von   
      \href{http://de.wikipedia.org/wiki/Dedekindscher_Schnitt}{\emph{Dedekindschen-Schnitten}}
      definieren.  Aus Zeitgr\"unden werden wir die Diskussion der Dedekindschen-Schnitte allerdings
      nur anrei\ss{}en k\"onnen.
\item Anschlie\ss{}end f\"uhren wir den f\"ur den Rest der Vorlesung grundlegenden Begriff des 
      \href{http://de.wikipedia.org/wiki/Grenzwert_(Folge)}{\emph{Grenzwerts}} einer
      \href{http://de.wikipedia.org/wiki/Folge_(Mathematik)}{\emph{Folge}} ein. 
      Wir werden beispielsweise sehen, dass die Folge $(b_1,b_2, b_3, \cdots)$, die induktiv durch
      \\[0.2cm]
      \hspace*{1.3cm}
      $b_1 := 1$ \quad und \quad $\ds b_{n+1} := \frac{1}{2} \cdot \left(b_n + \frac{2}{b_n}\right)$ f\"ur alle $n \in \mathbb{N}$
      \\[0.2cm]
      definiert ist, gegen die Zahl $\sqrt{2}$ \emph{konvergiert}.  Im Anschluss daran betrachten
      wir \href{https://de.wikipedia.org/wiki/Reihe_(Mathematik)}{\emph{Reihen}}.  Wir werden
      beispielsweise sehen, dass f\"ur reelle Zahlen $q$, f\"ur die $|q| < 1$ ist, die Gleichung
      \\[0.2cm]
      \hspace*{1.3cm}
      $\ds\sum\limits_{n=0}^{\infty} q^n = \frac{1}{1 - q}$
      \\[0.2cm]
      gilt.
\item Das vierte Kapitel diskutiert die Begriffe 
      \href{http://de.wikipedia.org/wiki/Stetigkeit}{\emph{Stetigkeit}} und 
      \href{http://de.wikipedia.org/wiki/Differenzierbarkeit}{\emph{Differenzierbarkeit}}.
\item Das f\"unfte Kapitel zeigt verschiedene Anwendungen der bis dahin dargestellten Theorie.
      Insbesondere werden 
      \href{http://de.wikipedia.org/wiki/Taylorreihe}{\emph{Taylor-Reihen}} diskutiert. Diese k\"onnen beispielsweise zur
      Berechnung der trigonometrischen Funktionen verwendet werden.  So werden wir beispielsweise
      sehen, dass wir f\"ur eine reelle Zahl $x$ den Wert $\sin(x)$ durch den Ausdruck
      \\[0.2cm]
      \hspace*{1.3cm}
      $\ds \sum\limits_{n=0}^{\infty} (-1)^n \cdot \frac{x^{2 \cdot n + 1}}{(2 \cdot n + 1)!} = x - \frac{1}{3!} \cdot x^3 + \frac{1}{5!} \cdot x^5 \mp \cdots$
      \\[0.2cm]
      berechnen k\"onnen.  Au\ss{}erdem diskutieren wir in diesem Kapitel Verfahren zur numerischen L\"osung
      von Gleichungen.  Beispielsweise zeigen wir, wie wir die Gleichung
      \\[0.2cm]
      \hspace*{1.3cm}
      $\cos(x) = x$ 
      \\[0.2cm]
      numerisch mit Hilfe einer
      \href{https://de.wikipedia.org/wiki/Fixpunktiteration}{Fixpunkt-Iteration} l\"osen k\"onnen. 
\item Das sechste Kapitel besch\"aftigt sich mit der
      \href{http://de.wikipedia.org/wiki/Integralrechnung}{\emph{Integralrechnung}}. 
\item Im siebten Kapitel zeigen wir, dass die Kreiszahl  \href{http://de.wikipedia.org/wiki/Kreiszahl}{$\pi$} 
      und die eulersche Zahl \href{http://de.wikipedia.org/wiki/Eulersche_Zahl}{$e$} keine rationalen Zahlen sind.
\item Im achten Kapitel diskutieren wir \href{http://de.wikipedia.org/wiki/Fourierreihe}{\emph{Fourier-Reihen}}.
      Unter anderem werden wir zeigen, dass
      \\[0.2cm]
      \hspace*{1.3cm}
      $\ds \sum\limits_{n=1}^{\infty} \frac{1}{n^2} = \frac{\pi^2}{6}$
      \\[0.2cm] 
      gilt.  Fourierreihen sind eine der Grundlage der Theorie der
      \href{http://de.wikipedia.org/wiki/Digitale_Signalverarbeitung}{\emph{digitalen Signalverarbeitung}}.
\item Das letzte Kapitel gibt einen kurzen Ausblick auf
      \href{http://de.wikipedia.org/wiki/Rundungsfehler}{\emph{Rundungs-Fehler}}. 
\end{enumerate}


\section{Ziel der Vorlesung}
Es ist nicht m\"oglich, im zeitlichen Rahmen der Vorlesung alle Aspekte der Analysis zu behandeln.
Insbesondere werden wir viele interessante Anwendungen der Analysis in der Informatik nicht
diskutieren k\"onnen.  Eine vollst\"andige Darstellung der Analysis ist aber auch gar nicht das Ziel,
das ich mir f\"ur diese Vorlesung gesetzt habe:  Mir geht es vor allem  darum, Ihnen die F\"ahigkeit zu
vermitteln, sich selbstst\"andig in mathematische Fachliteratur einarbeiten zu k\"onnen.  Dazu m\"ussen
Sie in der Lage 
sein, mathematische Beweise sowohl zu verstehen als auch selber entwickeln zu k\"onnen.   Dies ist ein
wesentlicher Unterschied zu der Mathematik, an die sich viele von Ihnen auf der Schule gew\"ohnt haben:
Dort werden prim\"ar Verfahren vermittelt, mit denen sich spezielle Probleme l\"osen lassen.  Die Kenntnis
solcher Verfahren ist allerdings in der Praxis nicht mehr so wichtig wie fr\"uher, denn heutzutage
werden solche Verfahren als Software-Pakete implementiert und daher besteht kaum mehr Bedarf daf\"ur,
diese Verfahren von Hand anzuwenden.  Statt dessen wird in
dieser Vorlesung der mathematische Beweis-Begriff im Vordergrund stehen.  Die Analysis dient uns
daher vor allem als ein Beispiel einer mathematischen Theorie, an Hand derer wir das mathematische 
Denken \"uben k\"onnen. 

\section{Notation}
In diesem Skript definieren wir die Menge der nat\"urlichen Zahlen $\mathbb{N}$ \"uber die Formel
\\[0.2cm]
\hspace*{1.3cm}
$\mathbb{N} := \{ 1, 2, 3, \cdots \}$.
\\[0.2cm]
Weiter definieren wir
\\[0.2cm]
\hspace*{1.3cm}
$\mathbb{N}_0 := \{ 0 \} \cup \mathbb{N}$.
\\[0.2cm]
Im \"ubrigen schlie\ss{}t dieses Skript an mein
\href{https://github.com/karlstroetmann/Lineare-Algebra/blob/master/Script/lineare-algebra.pdf}{Skript zur linearen Algebra} 
an und verwendet die selben Notationen, die ich dort zur Definition von Mengen eingef\"uhrt habe.
\pagebreak

\section{Eine Bitte}
Dieses Skript enth\"alt sicher noch den einen oder anderen Fehler.  Sollte Ihnen ein Fehler auffallen, so bitte ich
um einen Hinweis unter der Adresse
\\[0.2cm]
\hspace*{1.3cm}
\href{mailto:karl.stroetmann@dhbw-mannheim.de}{karl.stroetmann@dhbw-mannheim.de}.
\\[0.2cm]
Es bringt mir wenig, wenn Sie mich innerhalb meiner Vorlesung auf einen Tipp- oder
Rechtschreibfehler hinweisen, denn bis ich dazu komme, einen solchen Fehler zu korrigieren, habe ich
ihn meistens schon wieder vergessen.  Daher habe ich die Bitte, dass Sie mir etwaige Fehler per Email mitteilen.
Wenn Sie mit \href{http://github.com}{\texttt{github}} vertraut sind, k\"\"onnen Sie mir auch gerne einen
\href{https://help.github.com/articles/using-pull-requests}{\textsl{Pull Request}} schicken.


%%% Local Variables: 
%%% mode: latex
%%% TeX-master: "analysis"
%%% End: 
